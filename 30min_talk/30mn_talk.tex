\documentclass[unknownkeysallowed]{beamer}
\usepackage[utf8]{inputenc}
\usepackage[english]{babel}
\usepackage{dirtree}
\usepackage{tikz}
\usetikzlibrary{calc}
\usetikzlibrary{tikzmark}
\usepackage{../sty/beamer_js}
\usepackage{../sty/shortcuts_js}

\definecolor{tomcolor}{rgb}{0.,.3,.6}



\definecolor{kw}{RGB}{0,112,26}
\definecolor{var}{RGB}{0,133,182}
\definecolor{func}{RGB}{0,35,126}
\definecolor{palegreen}{RGB}{236,240,241}
\setbeamercolor{bgcolor}{fg=black!60,bg=palegreen}

\lstset{basicstyle=\color{tomcolor}\scriptsize\ttfamily}
\addbibresource{../biblio/biblio.bib}
\captionsetup[subfigure]{labelformat=empty}


\newcommand*{\colorboxed}{}
\def\colorboxed#1#{%
  \colorboxedAux{#1}%
}
\newcommand*{\colorboxedAux}[3]{%
  % #1: optional argument for color model
  % #2: color specification
  % #3: formula
  \begingroup
    \colorlet{cb@saved}{.}%
    \color#1{#2}%
    \boxed{%
      \color{cb@saved}%
      #3%
    }%
  \endgroup
}

\begin{document}

%%%%%%%%%%%%%%%%%%%%%%%%%%%%%%%%%%%%%%%%%%%%%%%%%%%%%%%%%%%%%%%%%%%%%%%%%%%%%%%
%%%%%%%%%%%%%%%%%%%%%%             Headers               %%%%%%%%%%%%%%%%%%%%%%
%%%%%%%%%%%%%%%%%%%%%%%%%%%%%%%%%%%%%%%%%%%%%%%%%%%%%%%%%%%%%%%%%%%%%%%%%%%%%%%


%%%%%%%%%%%%%%%%%%%%%%%%%%%%%%%%%%%%%%%%%%%%%%%%%%%%%%%%%%%%%%%%%%%%%%%%%%%%%%%
\begin{frame}
\bigskip
\bigskip
\begin{center}{
\LARGE\color{marron}
\textbf{\texttt{benchopt}:\\
		Benchmarking optimization Algorithms}
\textbf{ }\\
\vspace{0.5cm}
}

\color{marron}
% \textbf{or... an introduction to Beamer instead}
\end{center}

\vspace{0.5cm}

\begin{center}
\textbf{Zaccharie Ramzi} \\
\vspace{0.1cm}
\url{https://zaccharieramzi.fr/}\\
\vspace{0.5cm}
ENS Ulm - CNRS \\
\end{center}

\centering

\end{frame}
%%%%%%%%%%%%%%%%%%%%%%%%%%%%%%%%%%%%%%%%%%%%%%%%%%%%%%%%%%%%%%%%%%%%%%%%%%%%%%%


%%%%%%%%%%%%%%%%%%%%%%%%%%%%%%%%%%%%%%%%%%%%%%%%%%%%%%%%%%%%%%%%%%%%%%%%%%%%%%%
\begin{frame}[fragile]{The problem}

    \begin{columns}
        \column{.5\textwidth}
        \includegraphics[width=\textwidth]{tweet_steinner}
        \includegraphics[width=\textwidth]{tweet_steinner2}
        \column{.5\textwidth}
            \includegraphics[width=\textwidth]{DL_optimizers}
    \end{columns}

\end{frame}
%%%%%%%%%%%%%%%%%%%%%%%%%%%%%%%%%%%%%%%%%%%%%%%%%%%%%%%%%%%%%%%%%%%%%%%%%%%%%%%


%%%%%%%%%%%%%%%%%%%%%%%%%%%%%%%%%%%%%%%%%%%%%%%%%%%%%%%%%%%%%%%%%%%%%%%%%%%%%%%
\begin{frame}{Benchmarking algorithms in practice}

Choosing the best algorithm to solve an optimization problem often depends on:\\[1em]

\begin{minipage}{.8\textwidth}
\begin{itemize}
	\item the data \mybold{scale, conditionning};
	\item the objective parameters \mybold{regularisation};
	\item the implementation \mybold{complexity, language}.
\end{itemize}
\end{minipage}

\vskip2em
An impartial selection requires a time-consuming {\bf benchmark}!\\[1em]


The goal of {\bf \texttt{benchopt}} is to make this step as easy as possible.

\end{frame}
%%%%%%%%%%%%%%%%%%%%%%%%%%%%%%%%%%%%%%%%%%%%%%%%%%%%%%%%%%%%%%%%%%%%%%%%%%%%%%%



%%%%%%%%%%%%%%%%%%%%%%%%%%%%%%%%%%%%%%%%%%%%%%%%%%%%%%%%%%%%%%%%%%%%%%%%%%%%%%%
\begin{frame}[fragile]{\texttt{benchopt}}

Doing a  benchmark for the $\ell_2$ regularized logistic regression with multiple solvers and datasets is now as easy as calling:\\[1em]

\begin{lstlisting}[language=bash]
git clone https://github.com/benchopt/benchmark_logreg_l2
benchopt run ./benchmark_logreg_l2
\end{lstlisting}

\vskip1.5em
\includegraphics[width=.45\textwidth]{logreg_l2}
\hskip3ex
\includegraphics[width=.45\textwidth]{logreg_l2_1}

\end{frame}
%%%%%%%%%%%%%%%%%%%%%%%%%%%%%%%%%%%%%%%%%%%%%%%%%%%%%%%%%%%%%%%%%%%%%%%%%%%%%%%


%%%%%%%%%%%%%%%%%%%%%%%%%%%%%%%%%%%%%%%%%%%%%%%%%%%%%%%%%%%%%%%%%%%%%%%%%%%%%%%
\begin{frame}{\texttt{benchopt}}

    \texttt{benchopt} can also compare the same algo in different languages.\\[1em]

    Here is an example comparing PGD in: Python; R; Julia.\\[1em]
    {\centering
    \vskip.5em
    \includegraphics[width=.65\textwidth]{lasso_3_languages}\\}

\end{frame}
%%%%%%%%%%%%%%%%%%%%%%%%%%%%%%%%%%%%%%%%%%%%%%%%%%%%%%%%%%%%%%%%%%%%%%%%%%%%%%%


%%%%%%%%%%%%%%%%%%%%%%%%%%%%%%%%%%%%%%%%%%%%%%%%%%%%%%%%%%%%%%%%%%%%%%%%%%%%%%%
\begin{frame}{\texttt{benchopt}}

    \texttt{benchopt} also allow to publish easily benchmark results:\\[1em]


    {\centering

    \url{https://benchopt.github.io/results/}\\[2em]
    \vskip.5em
    \includegraphics[width=.65\textwidth]{benchopt_results}\\}

\end{frame}
%%%%%%%%%%%%%%%%%%%%%%%%%%%%%%%%%%%%%%%%%%%%%%%%%%%%%%%%%%%%%%%%%%%%%%%%%%%%%%%


%%%%%%%%%%%%%%%%%%%%%%%%%%%%%%%%%%%%%%%%%%%%%%%%%%%%%%%%%%%%%%%%%%%%%%%%%%%%%%%
\begin{frame}{What is \texttt{benchopt}?}
    \centering
    \texttt{benchopt} is a set of meta-benchmarking tools.
    It offers:
    \begin{itemize}
        \item a \mybold{template repository} to create benchmarks;
        \item a \mybold{Python library} to run benchmarks;
        \item a \mybold{web interface} to publish results.
    \end{itemize}
\end{frame}
%%%%%%%%%%%%%%%%%%%%%%%%%%%%%%%%%%%%%%%%%%%%%%%%%%%%%%%%%%%%%%%%%%%%%%%%%%%%%%%


%%%%%%%%%%%%%%%%%%%%%%%%%%%%%%%%%%%%%%%%%%%%%%%%%%%%%%%%%%%%%%%%%%%%%%%%%%%%%%%
\begin{frame}{\texttt{Benchmark}}
    \centering
    \includegraphics[width=.9\textwidth]{benchmark_lasso}\\
\end{frame}
%%%%%%%%%%%%%%%%%%%%%%%%%%%%%%%%%%%%%%%%%%%%%%%%%%%%%%%%%%%%%%%%%%%%%%%%%%%%%%%


%%%%%%%%%%%%%%%%%%%%%%%%%%%%%%%%%%%%%%%%%%%%%%%%%%%%%%%%%%%%%%%%%%%%%%%%%%%%%%%
\begin{frame}{Benchmark: principle}

A benchmark is a directory with:\\
\begin{itemize}
    \item an \lstinline+objective.py+ file with an \lstinline+Objective+;
    \item a directory \lstinline+solvers+ with one file per \lstinline+Solver+;
    \item a directory \lstinline+datasets+ with \lstinline+Dataset+ generators/fetchers.
\end{itemize}
\vskip1em

\includegraphics[width=.9\textwidth,trim={0 0 0 2.8em},clip]{benchopt_structure}

\vskip1em
The \lstinline+benchopt+ client runs a cross product and generates a csv file + convergence plots like above.
\end{frame}
%%%%%%%%%%%%%%%%%%%%%%%%%%%%%%%%%%%%%%%%%%%%%%%%%%%%%%%%%%%%%%%%%%%%%%%%%%%%%%%


%%%%%%%%%%%%%%%%%%%%%%%%%%%%%%%%%%%%%%%%%%%%%%%%%%%%%%%%%%%%%%%%%%%%%%%%%%%%%%%
\begin{frame}{Benchmark: Objective \& Dataset}

    \begin{beamercolorbox}[rounded=true,shadow=true,leftskip=2ex,colsep*=.75ex]{bgcolor}%
            \texttt{
                \textcolor{kw}{\bf class} \textcolor{var}{\bf Objective(BaseObjective)}:\\
                \hskip4ex name = \textcolor{var!60}{"Benchmark Name"}\\[1em]
                \hskip4ex\textcolor{kw}{\bf def} \textcolor{func}{set\_data}%
                (self, X, y):\\
                \hskip8ex\# Store data\\
                \hskip4ex\textcolor{kw}{\bf def} \textcolor{func}{compute}%
                (self, beta):\\
                \hskip8ex return dict\{obj1:.., obj2:..\}\\
                \hskip4ex\textcolor{kw}{\bf def} \textcolor{func}{to\_dict}%
                (self):\\
                \hskip8ex return dict\{X:.., y:.., reg:..\}\\
            }
    \end{beamercolorbox}%
    \vskip1.5em
    \begin{beamercolorbox}[rounded=true,shadow=true,leftskip=2ex,colsep*=.75ex]{bgcolor}%
        \texttt{
            \textcolor{kw}{\bf class} \textcolor{var}{\bf Dataset(BaseDataset)}:\\
            \hskip4ex name = \textcolor{var!60}{"Dataset Name"}\\[1em]
            \hskip4ex\textcolor{kw}{\bf def} \textcolor{func}{get\_data}%
            (self):\\
            \hskip8ex return dict\{X:.., y:..\}\\
        }
    \end{beamercolorbox}%

\end{frame}
%%%%%%%%%%%%%%%%%%%%%%%%%%%%%%%%%%%%%%%%%%%%%%%%%%%%%%%%%%%%%%%%%%%%%%%%%%%%%%%


%%%%%%%%%%%%%%%%%%%%%%%%%%%%%%%%%%%%%%%%%%%%%%%%%%%%%%%%%%%%%%%%%%%%%%%%%%%%%%%
\begin{frame}{Benchmark: Solver}

    \begin{beamercolorbox}[rounded=true,shadow=true,leftskip=2ex,colsep*=.75ex]{bgcolor}%
        \texttt{
            \textcolor{kw}{\bf class} \textcolor{var}{\bf Solver(BaseSolver)}:\\
            \hskip4ex name = \textcolor{var!60}{"Solver Name"}\\[1em]
            \hskip4ex\textcolor{kw}{\bf def} \textcolor{func}{set\_objective}%
            (self, X, y, reg):\\
            \hskip8ex\# Store objective info\\[1em]
            \hskip4ex\textcolor{kw}{\bf def} \textcolor{func}{run}%
            (self, n\_iter):\\
            \hskip8ex \# Run computations for \textcolor{var}{n\_iter}\\[1em]
            \hskip4ex\textcolor{kw}{\bf def} \textcolor{func}{get\_result}%
            (self):\\
            \hskip8ex return \textcolor{var}{beta}\\
        }
    \end{beamercolorbox}%

    \vskip1em
    \rem {\bf Flexible API}\\
    \begin{itemize}
        \item \texttt{\textcolor{func}{get\_data}} and \texttt{\textcolor{func}{set\_objective}} enable compatibility between packages.
        \item \texttt{\textcolor{var}{n\_iter}} can be replaced with a tolerance or a callback.
    \end{itemize}

\end{frame}
%%%%%%%%%%%%%%%%%%%%%%%%%%%%%%%%%%%%%%%%%%%%%%%%%%%%%%%%%%%%%%%%%%%%%%%%%%%%%%%

%%%%%%%%%%%%%%%%%%%%%%%%%%%%%%%%%%%%%%%%%%%%%%%%%%%%%%%%%%%%%%%%%%%%%%%%%%%%%%%
\begin{frame}{Contributions : two use cases}

    \begin{itemize}
        \item \mybold{Create} a new benchmark : for a problem that does not have a benchmark yet.
        \item Add a solver/dataset to an \mybold{existing} benchmark.
    \end{itemize}

\end{frame}
%%%%%%%%%%%%%%%%%%%%%%%%%%%%%%%%%%%%%%%%%%%%%%%%%%%%%%%%%%%%%%%%%%%%%%%%%%%%%%%


%%%%%%%%%%%%%%%%%%%%%%%%%%%%%%%%%%%%%%%%%%%%%%%%%%%%%%%%%%%%%%%%%%%%%%%%%%%%%%%
\begin{frame}{Practical implementation}
    \centering
    % in this slide I will have 2 columns
    % the one of the left will present a fake algorithm to solve an optimization problem
    % It will feature on top the fake optimization problem and in the bottom the fake algorithm
    % the column on the right will illustrate what each line in the algorithm corresponds to
    % in a typical benchopt benchmark
    % it will feature on top a tree structure of the files of the benchmark
    % and in the bottom the code corresponding to the part of the algorithm that is currently highlighted
    \noindent
        \begin{minipage}[t]{0.41\textwidth}
            \begin{equation*}
                \operatorname{argmin}_x L(x, D)
            \end{equation*}
            % then the fake algorithm in pseudo-code using algorithm2e
            \begin{algorithm}[H]
                \LinesNumbered
                \SetAlgoLined
                \KwIn{\alt<3>{\colorboxed{marron}{\text{data }D, \text{ loss }L}}{data \alt<1>{\colorboxed{marron}{D}}{$D$}, loss \alt<2>{\colorboxed{marron}{L}}{$L$}}}
                Set $x_0$\\
                \While{\alt<4>{\colorboxed{marron}{\text{not converged}}}{not converged}}{
                    \alt<5>{\colorboxed{marron}{x_n = x_{n-1}}}{$x_n = x_{n-1}$}
                }
                \Return{$x_n$}
                \caption{Solver 1}
            \end{algorithm}

        \end{minipage}
        \hfill
        \begin{minipage}[t]{0.57\textwidth}
            %on slide 1 we will have the tree structure with datasets/my_dataset.py and the corresponding code
            \only<1>{
            \dirtree{%
                .1 my\_benchmark.
                .2 datasets.
                .3 dataset1.py.
            }
            \begin{beamercolorbox}[rounded=true,shadow=true,leftskip=2ex,colsep*=.75ex]{bgcolor}%
                \texttt{
                    \textcolor{kw}{\bf def} \textcolor{func}{get\_data}%
                    (self):\\
                    \hskip8ex return dict\{D: D\}
                }
            \end{beamercolorbox}%
            \vskip1.5em
            }
            % on slide 2 we will have the tree structure with objective.py and the corresponding code
            \only<2>{
            \dirtree{%
                .1 my\_benchmark.
                .2 objective.py.
            }
            \begin{beamercolorbox}[rounded=true,shadow=true,leftskip=2ex,colsep*=.75ex]{bgcolor}%
                \texttt{
                    \textcolor{kw}{\bf def} \textcolor{func}{set\_data}%
                    (self, D):\\
                    \hskip8ex self.D = D\\
                    \hskip4ex\textcolor{kw}{\bf def} \textcolor{func}{compute}%
                    (self, x):\\
                    \hskip8ex return L(x, self.D)\\
                    \hskip4ex\textcolor{kw}{\bf def} \textcolor{func}{to\_dict}%
                    (self):\\
                    \hskip8ex return dict\{D=D, L=L\}
                }
        \end{beamercolorbox}%
            }
        % on slide 3 we will have the tree structure with solvers/my_solver.py and the corresponding code
        \only<3->{
            \dirtree{%
                .1 my\_benchmark.
                .2 solvers.
                .3 solver1.py.
            }
            \begin{beamercolorbox}[rounded=true,shadow=true,leftskip=2ex,colsep*=.75ex]{bgcolor}%
                \texttt{
                    \textcolor{kw}{\bf def} \textcolor{func}{set\_objective}%
                    (self, D, L):\\
                    \hskip8ex self.D = D\\
                    \hskip8ex self.L = L\\[1em]
                    \hskip4ex\textcolor{kw}{\bf def} \textcolor{func}{run}%
                    (self, callback):\\
                    \hskip8ex x = np.random.randn(...)\\
                    \hskip8ex \textcolor{kw}{\bf while} callback(x):\\
                    \hskip12ex x = x\\
                    \hskip8ex self.x = x\\[1em]
                    \hskip4ex\textcolor{kw}{\bf def} \textcolor{func}{get\_result}%
                    (self):\\
                    \hskip8ex return x
                }
            \end{beamercolorbox}%
        }
        \end{minipage}


\end{frame}
%%%%%%%%%%%%%%%%%%%%%%%%%%%%%%%%%%%%%%%%%%%%%%%%%%%%%%%%%%%%%%%%%%%%%%%%%%%%%%%

%%%%%%%%%%%%%%%%%%%%%%%%%%%%%%%%%%%%%%%%%%%%%%%%%%%%%%%%%%%%%%%%%%%%%%%%%%%%%%%
\begin{frame}{\texttt{benchopt}}
    \centering
    \includegraphics[width=.9\textwidth]{benchopt}\\
\end{frame}
%%%%%%%%%%%%%%%%%%%%%%%%%%%%%%%%%%%%%%%%%%%%%%%%%%%%%%%%%%%%%%%%%%%%%%%%%%%%%%%



%%%%%%%%%%%%%%%%%%%%%%%%%%%%%%%%%%%%%%%%%%%%%%%%%%%%%%%%%%%%%%%%%%%%%%%%%%%%%%%
\begin{frame}{\texttt{benchopt: Making tedious tasks easy}}

    {\bf Automating tasks:}\\[1.2em]
    \begin{itemize}\itemsep.7em
        \item Automatic installation of competitors solvers.
        \item Parametrized datasets, objectives and solvers and run on cross products.
        \item Make sure to quantify the variance.
        \item Automatic caching.
        \item First visualization of the results.
        \item Automatic parallelization (locally or on a cluster), ... ?
    \end{itemize}
\end{frame}
%%%%%%%%%%%%%%%%%%%%%%%%%%%%%%%%%%%%%%%%%%%%%%%%%%%%%%%%%%%%%%%%%%%%%%%%%%%%%%%

%%%%%%%%%%%%%%%%%%%%%%%%%%%%%%%%%%%%%%%%%%%%%%%%%%%%%%%%%%%%%%%%%%%%%%%%%%%%%%%
\begin{frame}{The intro paper}

    We have written a paper to introduce the project and the philosophy behind it.\\
    \vspace{1em}
    \cite{benchopt}\\
    \vspace{1em}
    We also highlight some examplary findings on different optimization problems similar to what you might encounter in your research.
\end{frame}
%%%%%%%%%%%%%%%%%%%%%%%%%%%%%%%%%%%%%%%%%%%%%%%%%%%%%%%%%%%%%%%%%%%%%%%%%%%%%%%


%%%%%%%%%%%%%%%%%%%%%%%%%%%%%%%%%%%%%%%%%%%%%%%%%%%%%%%%%%%%%%%%%%%%%%%%%%%%%%%
\begin{frame}{Credits}
    \begin{figure}
         \centering
         \hfill
         \begin{subfigure}[b]{0.188\textwidth}
             \centering
            \includegraphics[width=0.969\textwidth]{jsalmon}
            \caption{J. Salmon\\ INRIA Parietal}
         \end{subfigure}
         \hfill
         \begin{subfigure}[b]{0.188\textwidth}
             \centering
            \includegraphics[width=0.969\textwidth]{agramfort}
            \caption{A. Gramfort\\ INRIA Parietal}
         \end{subfigure}
         \hfill
         \begin{subfigure}[b]{0.188\textwidth}
             \centering
            \includegraphics[width=0.969\textwidth]{tommoral}
            \caption{T. Moreau\\ INRIA Parietal}
         \end{subfigure}\hfill\\[2em]
         \begin{subfigure}[b]{0.188\textwidth}
             \centering
             \includegraphics[width=0.969\textwidth]{nidham}
             \caption{N. Gazagnadou\\Telecom Paris}
         \end{subfigure}
         \hfill
         \begin{subfigure}[b]{0.188\textwidth}
             \centering
            \includegraphics[width=0.969\textwidth]{tanglef}
             \caption{T. Lefort\\Univ. Montpellier}
         \end{subfigure}
         \hfill
         \begin{subfigure}[b]{0.188\textwidth}
             \centering
             \includegraphics[width=0.969\textwidth]{mmassias}
             \caption{M. Massias\\Univ. of Genova }
         \end{subfigure}
         \hfill
         \begin{subfigure}[b]{0.188\textwidth}
             \centering
            \includegraphics[width=0.969\textwidth]{tomdlt}
            \caption{T. Dupré la Tour\\ UC Berkeley}
         \end{subfigure}
    \end{figure}
\end{frame}
%%%%%%%%%%%%%%%%%%%%%%%%%%%%%%%%%%%%%%%%%%%%%%%%%%%%%%%%%%%%%%%%%%%%%%%%%%%%%%%




% %%%%%%%%%%%%%%%%%%%%%%%%%%%%%%%%%%%%%%%%%%%%%%%%%%%%%%%%%%%%%%%%%%%%%%%%%%%%%%%
% % Uncomment for references
% \begin{frame}{Bibliographie}
% \printbibliography
% \end{frame}
%  %%%%%%%%%%%%%%%%%%%%%%%%%%%%%%%%%%%%%%%%%%%%%%%%%%%%%%%%%%%%%%%%%%%%%%%%%%%%%%



\end{document}
